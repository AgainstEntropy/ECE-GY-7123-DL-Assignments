\noindent \textred{2.} For the following array: $A =\{ 3, 9, 5, 8, 15, 7, 4, 10, 6, 12, 16 \}$,
\begin{enumerate}
    \item[(a)] Create a max heap using the algorithm BUILD-MAX-HEAP. \\
    \textblue{
    We first create a heap from the array $A$ as left below: \\
    \begin{tikzpicture}[heap]
          \node {3}
          child{node{9}
            child{node{8} child{node{10}} child{node{6}}} 
            child{node{15} child{node{12}} child{node{16}}}}
          child{node{5} child{node{7}} child{node{4}}}
          ;
        % \draw (0, 0) 
        % .. controls ++(165:-1) and ++(240: 1) .. ( 3, 2)
        % .. controls ++(240:-1) and ++(165:-1) .. ( 2, 4)
        % .. controls ++(165: 1) and ++(175:-2) .. (-1, 2)
        % .. controls ++(175: 2) and ++(165: 1) .. ( 0, 0);
    \end{tikzpicture}
    \hspace{10pt}
    \begin{tikzpicture}[heap]
          \node {16}
          child{node{15}
            child{node{10} child{node{8}} child{node{6}}} 
            child{node{12} child{node{3}} child{node{9}}}}
          child{node{7} child{node{5}} child{node{4}}}
          ;
    \end{tikzpicture} \\
    After calling MAX-HEAPIFY on the first half elements, i.e., \{15, 8, 5, 9, 3\} one after another, the heap becomes upper right, which is \{16, 15, 7, 10, 12, 5, 4, 8, 6, 3, 9\} in the array form.
    }
    \item[(b)] Remove the largest item from the max heap you created in 2(a), using the HEAP-EXTRACT-MAX function. Show the array after you have removed the largest item. \\
    \textblue{
    After exchange the last element with the max one and extract the max value, the heap becomes left below: \\
    \begin{tikzpicture}[heap]
          \node {9}
          child{node{15}
            child{node{10} child{node{8}} child{node{6}}} 
            child{node{12} child{node{3}} child{node[dashed]{16}}}}
          child{node{7} child{node{5}} child{node{4}}}
          ;
    \end{tikzpicture} 
    \hspace{20pt}
    \begin{tikzpicture}[heap]
          \node{15}
          child{node{12}
            child{node{10} child{node{8}} child{node{6}}} 
            child{node{9} child{node{3}} child{node[dashed]{16}}}}
          child{node{7} child{node{5}} child{node{4}}}
          ;
    \end{tikzpicture} \\
    Then we call MAX-HEAPIFY on the new root, which turns the heap to upper right. In the array form, the heap is \{15, 12, 7, 10, 9, 5, 4, 8, 6, 3\}.
    }
    \item[(c)] Using the algorithm MAX-HEAP-INSERT, insert 11 into the heap that resulted from question 2(b). Show the array after insertion. \\
    \textblue{
    After append the array with $-\infty$, we call HEAP-INCERASE-KEY on the new element to increase its key to 11. Before recurrently exchange the last element with its parent, the heap is like left below \\
    \begin{tikzpicture}[heap]
          \node{15}
          child{node{12}
            child{node{10} child{node{8}} child{node{6}}} 
            child{node{9} child{node{3}} child{node{11}}}}
          child{node{7} child{node{5}} child{node{4}}}
          ;
    \end{tikzpicture} 
    \hspace{20pt}
    \begin{tikzpicture}[heap]
          \node{15}
          child{node{12}
            child{node{10} child{node{8}} child{node{6}}} 
            child{node{11} child{node{3}} child{node{9}}}}
          child{node{7} child{node{5}} child{node{4}}}
          ;
    \end{tikzpicture} \\
    After the HEAP-INCERASE-KEY operation finishes, the heap is \{15, 12, 7, 10, 11, 5, 4, 8, 6, 3, 9\} in the array form. \\
    }
\end{enumerate}
